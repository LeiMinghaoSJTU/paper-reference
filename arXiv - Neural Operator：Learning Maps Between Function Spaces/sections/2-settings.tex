\newpage
\section{Learning Operators(算子学习设定)}
\label{sec:setting}
在子章节~\ref{sec:genericPDE}中,我们描述了偏微分方程(PDEs)的通用设定,以便使后续讨论的背景更加具体。
在子章节~\ref{sec:PS}中,我们概述了算子学习的一般问题以及我们解决该问题的方法。
在子章节~\ref{sec:discretization}中,我们讨论了可用的函数型数据以及我们如何在数值上处理这些数据。


\vspace{1cm}
\subsection{Generic Parametric PDEs (泛型参数化偏微分方程)}
\label{sec:genericPDE}

我们考虑如下形式的一类通用偏微分方程(PDE, Partial Differential Equation):
\begin{align}
\label{eq:generalpde0}
\begin{split}
(\LL_a u)(x) &= f(x), \qquad x \in D, \\
u(x) &= 0, \qquad \quad \:\: x \in \partial D,
\end{split}
\end{align}
其中 $a \in \A$,$f \in \U^*$,$D \subset \R^d$ 是一个有界区域(bounded domain)。我们假设解 $u: D \to \R$ 属于Banach空间 $\U$,且 $\LL_{a}: \A \to \mathcal{L}(\U; \U^*)$ 是一个从参数Banach空间 $\A$ 映射到将 $\U$ 映射至其对偶空间 $\U^*$ 的(可能是无界的)线性算子空间 $\mathcal{L}(\U; \U^*)$ 的映射。由此PDE自然导出的一个算子是 $\textcolor{blue}{\G^\dagger:A \to \U,a\mapsto \LL_a^{-1}f=:u}$,它将参数映射到解,即 $a \mapsto u$。我们在第 \ref{ssec:darcy} 节中进一步研究的一个简单例子是:当 $\LL_a$ 是二阶椭圆算子 $-\nabla \cdot (a \nabla)$ 在齐次Dirichlet边界条件下的弱形式(weak form)时。在此设定下,$\A = L^\infty(D;\R_+)$,$\U = H^1_0(D;\R)$,以及 $\U^* = H^{-1}(D;\R)$。若有必要,我们将假设区域 $D$ 被离散化为 $K \in \N$ 个点,并且我们观测到 $N \in \N$ 组系数函数与(近似)解函数的配对数据 $\{a^{(i)}, u^{(i)}\}_{i=1}^N$,这些数据将用于模型训练(见第 \ref{sec:PS} 节)。我们假设 $a^{(i)}$ 是从支撑在 $\A$ 上的概率测度 $\mu$ 中独立同分布(i.i.d., independent and identically distributed)采样的样本,而 $u^{(i)}$ 是在 $\G^\dagger$ 下的前推(pushforward)像。


\vspace{1cm}

\subsection{问题设定}
\label{sec:PS}

% {\bf State the problem as supervised learning between Banach spaces of
% functions on a bounded domain $D$ in $\R^d$. Mapping function $a$
% to function $u$. I guess I'd be happy 
% to stick to the setting where input and output share the same domain. 
% (Maybe discuss how to generalize at the end of the paper). No reference 
% to any specific PDE here.}

我们的目标是通过使用有限数量的输入-输出对观测数据,来学习两个无限维空间之间的映射。我们将此问题具体化为以下设定。设 $\A$ 和 $\U$ 分别为定义在有界区域 $D \subset \R^d$、$D' \subset \R^{d'}$ 上的函数构成的巴拿赫空间(Banach spaces),且 $\Ftrue : \A \to \U$ 是一个(通常是)非线性映射(non-linear map)。假设我们有一组观测数据 $\{a^{(i)}, u^{(i)}\}_{i=1}^N$,其中 $a^{(i)} \sim \mu$ 是从支撑在 $\A$ 上的某个概率测度(probability measure)$\mu$ 中独立同分布(i.i.d.)抽取的样本,而 $u^{(i)} = \Ftrue(a^{(i)})$ 可能受到噪声污染。我们的目标是通过构造一个参数化的映射
\begin{equation}
\label{eq:approxmap2}
\F_{\theta} : \A \to \U, \quad \theta \in \R^p
\end{equation}
其中参数来自有限维空间 $\R^p$,然后选择适当的参数 $\theta^\dagger \in \R^p$,使得 $\F_{\theta^\dagger} \approx \Ftrue$。

我们希望在关于 $\mu$ 的平均意义下控制近似误差。特别地,假设 $\G^\dagger$ 是 $\mu$-可测的($\mu$-measurable),我们旨在控制近似在 $L^2_\mu(\A;\U)$ Bochner 范数下的误差:
\begin{equation}
\label{eq:bochner_error}
\|\G^\dagger - \G_\theta\|^2_{L^2_\mu(\A;\U)} = \E_{a \sim \mu} \|\G^\dagger(a) - \G_\theta(a)\|_{\U}^2 = \int_\A \|\G^\dagger(a) - \G_\theta(a)\|_{\U}^2 \: d \mu (a).
\end{equation}
这是一个自然的无限维学习框架,因为可以求解相应的经验风险最小化问题(empirical-risk minimization problem):
\begin{equation}
\label{eq:empirical_risk}
\min_{\theta \in \R^p} \E_{a \sim \mu} \|\G^\dagger(a) - \G_\theta(a)\|_{\U}^2 \approx \min_{\theta \in \R^p} \frac{1}{N} \sum_{i=1}^N \|u^{(i)} - \G_\theta(a^{(i)})\|_{\U}^2
\end{equation}
这直接平行于经典的有限维情形 \citep{Vapnik1998}。
\nk{除了使用 Bochner 范数度量误差外,我们还将考虑在 $\A$ 的紧集上一致度量误差的情形。具体而言,对于任意紧集 $K \subset \A$,我们考虑
\begin{equation}
    \label{eq:unform_risk}
    \sup_{a \in K} \|\G^\dagger (a) - \G_\theta (a) \|_\U
\end{equation}
这是逼近论(approximation theory)文献中更标准的误差度量方式。事实上,经典的神经网络逼近理论正是以类似于公式 \eqref{eq:unform_risk} 的形式表述的 \citep{hornik1989multilayer}。}


在第~\ref{sec:approximation}节中,我们将证明,对于我们所提出的网络结构,只要给定任意期望的误差容限(error tolerance),就存在一个参数维数 $p \in \N$ 以及相应的参数 $\theta^\dagger \in \R^p$,使得损失 \eqref{eq:bochner_error} 或 \eqref{eq:unform_risk} 小于指定的容限。然而,我们并未解决以下两个具有挑战性的开放问题:(a) 在固定参数维度 $p$ 下的误差刻画,或 (b) 在固定训练样本数量 $N$ 下的误差刻画。相反,我们在经验性的训练-测试(empirical test-train)框架下进行研究,即基于一个固定的训练集最小化 \eqref{eq:empirical_risk},并通过训练过程中未见过的新样本去近似估计 \eqref{eq:bochner_error}。

由于我们将方法论建立在无限维(infinite-dimensional)框架下,所有有限维近似都可以共享一组在(无近似的)无限维设定中定义的公共网络参数。特别地,我们的网络结构不依赖于函数 $a^{(i)},u^{(i)}$ 的离散化方式。

本文中使用的符号表示,以及一个有用的汇总表格,可在附录 \ref{sec:appendix_notation} 中找到。

\vspace{1cm}


\subsection{Discretization(离散化)}
\label{sec:discretization}

由于我们的数据 $a^{(i)}$ 和 $u^{(i)}$ 通常是函数,为了进行数值计算,我们假设只能访问它们在某些点上的逐点(point-wise)取值。为了说明这一点,我们将继续上一段的例子。为简便起见,假设 $D = D'$,并设输入和输出函数均为实值函数。令 $D^{(i)} = \{x^{(i)}_{\ell}\}_{\ell=1}^L \subset D$ 为区域 $D$ 的一个包含 $L$ 个点的离散化(discretization),并假设我们观测到了一组有限的输入-输出对(以 $j$ 索引)的值 $a^{(i)}|_{D^{(i)}}, u^{(i)}|_{D^{(i)}} \in \R^{L}$。

在下一节中,我们将提出一种受核方法启发的图神经网络(graph neural network)结构,该结构虽然在离散化数据上进行训练,但给定一个来自分布 $\mu$ 的输入 $a$,即可输出任意位置 $x \in D$ 处的解 $u(x)$。\nk{特别地,我们的离散化架构映射到空间 $\U$ 本身,而不是其离散化版本。此外,我们的参数化算子类是具有一致性(consistent)的:给定一组固定的参数,当输入的离散化被不断细化时,其结果会收敛到真实的函数空间算子。我们将在下文中精确阐述这一概念,并将具备此性质的架构称为函数空间架构(function space architectures)、网格不变架构(mesh-invariant architectures)或\textit{离散化不变}(discretization-invariant)架构。
\footnote{\nk{请注意,下文定义中集合 $D_{\bullet}$ 的索引方式与本段前文所用的不同。}}}

\begin{definition}
我们称任何一组嵌套集合序列 $D_1 \subset D_2 \subset \dots \subset D$为域 $D \subset \R^d$ 的一个\textcolor{red}{离散细化}(discrete refinement),如果 $|D_L| = L$ 对于任意 $L \in \N$ 成立,并且对于任意 $\epsilon > 0$,存在一个数 $L = L(\epsilon) \in \N$,使得
\[D \subseteq \bigcup_{x \in D_L} \{y \in \R^d : \|y - x\|_2 < \epsilon\}.\]
\end{definition}

\begin{definition}
给定域 $D \subset \R^d$ 的一个离散细化 $(D_L)_{L=1}^\infty$,其中任意一个成员 $D_L$ 被称为 $D$ 的一个\textcolor{red}{离散化}(discretization)。
\end{definition}



由于 $a:D \subset \R^d \to \R^m$,在 $L$ 个点上对该函数进行逐点求值(离散化)会得到数据集 $\{(x_{\ell}, a(x_{\ell}))\}_{\ell=1}^L.$ 注意,这可以被视为 $\R^{Ld} \times \R^{Lm}$ 中的一个向量。图 \ref{fig:discretization-invariance} 给出了一个网格细化(mesh refinement)的例子。
%%%%%%%%
\begin{definition}
假设 $\A$ 是定义在区域 $D \subset \R^d$ 上的取值于 $\R^m$ 的函数构成的巴拿赫空间(Banach space)。令 $\G : \A \to \U$ 是一个算子,$D_L$ 是 $D$ 的一个包含 $L$ 个点的离散化(discretization),且 $\hat{\G} : \R^{Ld} \times \R^{Lm} \to \U$ 为某个映射。对于任意紧集 $K \subset \A$,我们定义\textcolor{red}{离散化的一致风险}(discretized uniform risk)为
\[R_K (\G,\hat{\G},D_L) = \sup_{a \in K} \|\hat{\G}(D_L,a|_{D_L}) - \G(a) \|_{\U}.\]
\end{definition}
%%%%%%%%
%%%%%%%%

\begin{definition}\label{def:discretization_invariance}
设 $\Theta \subseteq \R^p$ 为一个有限维参数空间,$\G : \A \times \Theta \to \U$ 是一个参数化算子类(parametric class of operators)【比如说含参数$\theta$的神经算子族】,其参数为 $\theta \in \Theta$。给定区域 $D \subset \R^d$ 的一个离散细化(discrete refinement)$(D_n)_{n=1}^\infty$,我们称 $\G$ 是\textcolor{red}{离散化不变}(discretization-invariant)的,如果存在一系列映射 $\hat{\G}_1, \hat{\G}_2, \dots$,其中 $\hat{\G}_L : \R^{Ld} \times \R^{Lm} \times \Theta \to \U$,使得对于任意 $\theta \in \Theta$ 和任意紧集 $K \subset \A$,
\[\lim_{L \to \infty} R_K(\G(\cdot,\theta), \hat{\G}_L(\cdot,\cdot,\theta),D_L) = 0.\]
\end{definition}
我们证明了在第~\ref{sec:neuraloperators}节中提出的架构是离散化不变的。% in Section~\ref{sec:discritizational_invariance}.
我们还通过数值实验验证了这一性质,即当网格被不断细化时,近似误差近似保持不变。


% 这种性质非常理想,因为它允许使用一个具有固定参数数量的单一架构,在不同网格几何形状和不同离散化规模之间转移解。

% 这一理想特性是传统神经网络所不具备的,传统神经网络是有限维空间之间的映射,而最近提出的 DeepONet 模型则是从有限维到无限维空间的映射。在本研究中,我们提出了一种改进的 DeepONet 架构,即 DeepONet-Operator,它是一种全栈神经算子(full-stack neural operators),从而得到一个具有离散化不变性质的模型。本文提出的离散化不变性的定义是一个通用定义,当传统的神经网络(如 CNNs)配合输入端的网格插值(grid interpolation)时,也可被视为离散化不变模型。在输出端增加一个插值层可使这些模型成为无限维空间之间的映射。


这种性质非常理想,因为它允许使用一个具有固定参数数量的单一架构,在不同网格几何形状和不同离散化规模之间转移解。

这一理想特性是传统神经网络所不具备的,传统神经网络是有限维空间之间的映射(maps between finite-dimensional spaces),而最近提出的 DeepONet 模型则是从有限维到无限维空间的映射(maps from finite-dimensional to infinite-dimensional spaces)。在本研究中,我们提出了一种改进的 DeepONet 架构,称为 DeepONet-Operator,它是一种全栈神经算子(full-stack neural operators),从而得到一个具有离散化不变性(discretization invariance)的模型。本文提出的离散化不变性的定义是一个通用定义,当传统的神经网络(如 CNNs)配合输入端的网格插值(grid interpolation)时,也可被视为离散化不变模型。在输出端增加一个插值层可使这些模型成为无限维空间之间的映射。

我们注意到,虽然我们方法的应用基于函数的逐点取值(point-wise evaluations),但并不局限于这种情况。例如,人们也可以将函数用一组有限的截断基函数系数(truncated basis coefficients)来数值表示。此时,表示的不变性就体现在该系数集合的大小上。原则上,可以通过设计合适的架构来修改我们的方法以适应这种情况。在当前工作中,我们并未探讨这一方向。从神经算子(neural operators)的构造来看,当输入和输出函数在固定的网格上被求值时,这些固定网格上的神经算子架构与一类神经网络是重合的。