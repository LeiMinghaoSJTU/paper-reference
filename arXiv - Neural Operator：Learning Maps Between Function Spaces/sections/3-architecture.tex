\newpage
\section{Neural Operators(神经算子)}
\label{sec:neuraloperators}
% {\bf State the proposed architecture in a general form.
% No reference to any specific PDE.}

% \subsection{Neural Networks}
在本节中,我们概述神经算子(neural operator)框架。
我们假设输入函数 $a \in \A$ 是取值于 $\R^{d_a}$ 且定义在有界区域 $D \subset \R^d$ 上的函数,而输出函数 $u \in \U$ 是取值于 $\R^{d_u}$ 且定义在有界区域 $D' \subset \R^{d'}$ 上的函数。
所提出的架构 $\F_{\theta} : \A \to \U$ 具有以下整体结构:

\begin{enumerate}
    
    \item \textbf{提升}(Lifting):通过一个逐点函数(pointwise function)$\R^{d_a} \to \R^{d_{v_0}}$,将输入 $\{a: D \to \R^{d_a}\}$ 映射到其第一个隐藏表示 $\{v_0: D \to \R^{d_{v_0}}\}$。通常,我们选择 $d_{v_0} > d_a$,因此这是一个由完全局部算子(fully local operator)执行的提升操作。
    
    \item \textbf{迭代核积分}(Iterative Kernel Integration):对于 $t=0,\dots,T-1$,通过一个局部线性算子、一个非局部积分核算子(non-local integral kernel operator)和一个偏置函数之和的作用,将每一层的隐藏表示映射到下一层:
    $$\color{red}{\{v_t: D_t \to \R^{d_{v_t}}\} \mapsto \{v_{t+1}: D_{t+1} \to \R^{d_{v_{t+1}}}\}},$$
    并将其和与一个固定的逐点非线性函数(pointwise nonlinearity)进行复合。这里我们设定 $D_0 = D$ 且 $D_T = D'$,并要求 $\textcolor{red}{D_t \subset \R^{d_t}}$ 是一个有界区域。\footnote{\nk{此处集合 $D_{\bullet}$ 的索引方式与第 \ref{sec:discretization} 小节中之前使用的两种索引方式不同。索引 $t$ 并非物理时间,而是模型架构中的迭代(层)索引。}}
    
    \item \textbf{投影}(Projection):通过一个逐点函数 $\R^{d_{v_T}} \to \R^{d_u}$,将最后一个隐藏表示 $\{v_T: D' \to \R^{d_{v_T}}\}$ 映射到输出函数 $\{u: D' \to \R^{d_u}\}$。类似于第一步,我们通常选择 $d_{v_T} > d_u$,因此这是一个由完全局部算子执行的投影步骤。
\end{enumerate}


所概述的结构模仿了有限维神经网络的结构,其中隐藏表示被逐层映射以产生最终输出。具体而言,我们有
\begin{equation}
\label{eq:F}
    \cF_{\theta} \coloneqq \cQ \circ \sigma_T(W_{T-1} + \cK_{T-1} + b_{T-1}) \circ \cdots \circ \sigma_1(W_0 + \cK_0 + b_0) \circ \cP
\end{equation}
其中 $\cP: \R^{d_a} \to \R^{d_{v_0}}$ 和 $\cQ: \R^{d_{v_T}} \to \R^{d_u}$ 分别是局部提升(lifting)和投影(projection)映射,$W_t \in \R^{d_{v_{t+1}} \times d_{v_t}}$ 是局部线性算子(矩阵),$\cK_t: \{v_t: D_t \to \R^{d_{v_t}}\} \to \{v_{t+1}: D_{t+1} \to \R^{d_{v_{t+1}}}\}$ 是积分核算子(integral kernel operators),$b_t : D_{t+1} \to \R^{d_{v_{t+1}}}$ 是偏置函数(bias functions),而 $\sigma_t$ 是固定的激活函数,在每一层中作为逐点映射  $\textcolor{blue}{\R^{d_{v_{t+1}}} \to \R^{d_{v_{t+1}}}}$作用。输出维度 $d_{v_0},\dots,d_{v_T}$ 以及输入维度 $d_1,\dots,d_{T-1}$ 和定义域 $D_1,\dots,D_{T-1}$ 都是该架构的超参数(hyperparameters)。所谓局部映射,是指\underline{其作用是逐点的};特别地,对于提升和投影映射,我们有对任意 $x \in D$,$(\cP(a))(x) = \cP(a(x))$,以及对任意 $x \in D'$,$(\cQ(v_T))(x) = \cQ(v_T(x))$;类似地,对于激活函数,对任意 $x \in D_{t+1}$,$(\sigma(v_{t+1}))(x) = \sigma(v_{t+1}(x))$。
因此,当映射 $\cP$、$\cQ$ 和 $\sigma_t$ 的每个分量都被假定为博雷尔可测(Borel measurable)时,它们可以被视为定义了 Nemitskiy 算子 \citep[第6,7章]{dudley2011concrete}。这种解释使得我们能够在空间 $\A$ 或 $\U$ 中逐点求值没有良好定义时(例如当它们是勒贝格空间(Lebesgue)、索伯列夫空间(Sobolev)或Besov空间时)定义一般的神经算子架构。

所提出的架构 \eqref{eq:F} 与标准前馈神经网络的关键区别在于,所有操作都是直接在函数空间中定义的(\nk{注意到激活函数 $\cP$ 和 $\cQ$ 都通过其扩展为 Nemitskiy 算子来解释}),因此不依赖于数据的任何离散化。直观上,提升步骤将数据局部地映射到一个新空间,在该空间中 $\G^\dagger$ 的非局部部分更容易捕捉。\nk{我们在第~\ref{sec:numerics}节中通过数值实验验证了这一直觉;然而需要注意的是,对于第~\ref{sec:approximation}节中提出的理论,$\cP$ 取为恒等映射就足够了。} $\G^\dagger$ 的非局部部分随后通过由积分核算子与局部非线性函数复合而成的操作进行逐层逼近来学习。每个积分核算子是标准前馈网络中权重矩阵的函数空间对应物,因为它们是将一个函数空间映射到另一个函数空间的无限维线性算子。我们将通常为向量的偏置项推广为函数,并借鉴 ResNet 架构 \citep{he2016deep} 的思想,在应用非线性之前,进一步添加一个作用于前一层输出的局部线性算子。最后的投影步骤只是将结果映射回输出函数所在的空间。我们将 $\cP$、$\cQ$、$\{b_t\}$(通常本身是浅层神经网络)的参数、表示 $\{\cK_t\}$ 的核的参数(通常也是浅层神经网络)以及矩阵 $\{W_t\}$ 的参数合并到 $\theta \in \R^p$ 中。然而我们注意到,我们的框架是通用的,也可以采用其他参数化方式,例如多项式。

\vspace{1cm}
\subsection{积分核算子(Integral Kernel Operators)} 

我们定义三种在 \eqref{eq:F} 中使用的积分核算子 $\cK_t$。第一种情形下,设 $\kappa^{(t)} \in C(D_{t+1} \times D_t; \R^{d_{v_{t+1}} \times d_{v_t}})$,且 $\nu_t$ 为 $D_t$ 上的博雷尔测度(Borel measure)。于是我们定义 $\cK_t$ 为
\begin{equation}
    \label{eq:kernelop1}
    (\cK_t(v_t))(x) = \int_{D_t} \kappa^{(t)} (x,y) v_t(y) \: \text{d}\nu_t(y)
    \qquad \forall x \in D_{t+1}.
\end{equation}
%其中 $\nu_t$ 是 $D_t$ 上的博雷尔测度。
通常情况下,我们取 $\nu_t$ 为 $\R^{d_t}$ 上的勒贝格测度(Lebesgue measure),但正如第~\ref{sec:four_schemes} 节所讨论的那样,其他选择可用于加速计算,或通过嵌入先验信息(\textit{a priori} information)来辅助学习过程。式 \eqref{eq:kernelop1} 中定义的积分核算子构成了神经算子(neural operator)的基本形式,也是我们在第~\ref{sec:approximation} 节中进行理论分析、并在第~\ref{sec:numerics} 节数值实验中重点研究的形式。

第二种情形下,设 $\kappa^{(t)} \in C(D_{t+1} \times D_t \times \R^{d_a} \times \R^{d_a}; \R^{d_{v_{t+1}} \times d_{v_t}})$。于是我们定义 $\cK_t$ 为
\begin{equation}
    \label{eq:kernelop2}
    (\cK_t(v_t))(x) = \int_{D_t} \kappa^{(t)} (x,y,a(\Pi_{t+1}^D(x)),a(\Pi_t^D(y))) v_t(y) \: \text{d}\nu_t(y)
    \qquad \forall x \in D_{t+1}.
\end{equation}
其中 $\textcolor{red}{\Pi_t : D_{t+1} \to D_t}$ 是固定的\textbf{投影映射}。我们通过数值实验发现,对于某些偏微分方程(PDE)问题,形式 \eqref{eq:kernelop2} 的表现\textbf{优于} \eqref{eq:kernelop1},这是因为\underline{解 $u$ 对参数 $a$ 有很强的依赖性},
\nk{例如,\ref{ssec:DF} 小节中考虑的达西流(Darcy flow)问题}。事实上,若我们将 \eqref{eq:F} 视为一个离散时间动力系统,则输入 $a \in \A$ 仅通过初始条件进入系统,因此其影响会随着网络层数的增加而逐渐减弱。通过在核函数中直接引入对 $a$ 的依赖,我们确保该参数在整个网络架构中持续发挥作用。

最后一种情形下,设 $\kappa^{(t)} \in C(D_{t+1} \times D_t \times \R^{d_{v_t}} \times \R^{d_{v_t}}; \R^{d_{v_{t+1}} \times d_{v_t}})$。于是我们定义 $\cK_t$ 为
\begin{equation}
    \label{eq:kernelop3}
    (\cK_t(v_t))(x) = \int_{D_t} \kappa^{(t)} (x,y,v_t(\Pi_t(x)),v_t(y)) v_t(y) \: \text{d}\nu_t(y)
    \qquad \forall x \in D_{t+1}.
\end{equation}
其中 $\Pi_t : D_{t+1} \to D_t$ 是固定的映射。注意,与 \eqref{eq:kernelop1} 和 \eqref{eq:kernelop2} 不同,积分算子 \eqref{eq:kernelop3} 是非线性的,因为其核函数可以依赖于输入函数 $v_t$。借助这一定义,并结合对核函数 $\kappa_t$ 和测度 $\nu_t$ 的特定选择,我们在第~\ref{sec:transformers} 节中证明:神经算子是流行变换器架构(transformer architecture)\citep{vaswani2017attention} 在连续输入/输出空间上的推广。

\paragraph{Integral Kernel Operators} We define three version of the integral kernel operator \(\cK_t\) used in \eqref{eq:F}. For the first, let \(\kappa^{(t)} \in C(D_{t+1} \times D_t; \R^{d_{v_{t+1}} \times d_{v_t}})\) and let \(\nu_t\) be a Borel measure on \(D_t\). Then we define \(\cK_t\) by
\begin{equation}
    \label{eq:kernelop1}
    (\cK_t(v_t))(x) = \int_{D_t} \kappa^{(t)} (x,y) v_t(y) \: \text{d}\nu_t(y)
    \qquad \forall x \in D_{t+1}.
\end{equation}

\vspace{1cm}
\subsection{单隐藏层构造 (Single Hidden Layer Construction)}

在定义了积分核算子(integral kernel operator)的若干可能选择之后,我们现在可以明确写出由 \eqref{eq:F} 所定义架构中的完整一层。为简便起见,我们选取由 \eqref{eq:kernelop1} 给出的积分核算子,但需注意,其他定义 \eqref{eq:kernelop2}、\eqref{eq:kernelop3} 的处理方式类似。此时,单个隐藏层的更新可表示为
\begin{equation}
    \label{eq:onelayer}
    v_{t+1}(x) = \sigma_{t+1} \left ( W_t v_t( \Pi_{t} (x)) + \int_{D_t} \kappa^{(t)} (x,y) v_t(y) \: \text{d}\nu_t(y) + b_t(x) \right ) \qquad \forall x \in D_{t+1},
\end{equation}
其中 $\Pi_t : D_{t+1} \to D_t$ 是固定的映射。我们指出,由于我们通常考虑定义在同一区域上的函数,因此通常取 $\Pi_t$ 为恒等映射(identity)。

接下来,我们给出一个完整的单隐藏层架构(即 $T=2$)的例子。我们选取 $D_1 = D$,令 $\sigma_2$ 为恒等映射,并将 $\sigma_1$ 简记为 $\sigma$,假设其为任意激活函数。此外,为简化起见,我们设 $W_1 = 0$、$b_1 = 0$,并假设 $\nu_0 = \nu_1$ 为 $\R^d$ 上的勒贝格测度(Lebesgue measure)。于是 \eqref{eq:F} 变为
\begin{equation}
    \label{eq:singlehiddenlayer}
    (\G_\theta(a))(x) = \cQ \left (\int_D \kappa^{(1)}(x,y) \sigma \left ( W_0 \cP(a(y)) + \int_D \kappa^{(0)}(y,z) \cP (a(z)) \: \text{d}z + b_0(y) \right ) \: \text{d}y \right)
\end{equation}
对任意 $x \in D'$ 成立。在此例中,$\cP \in C(\R^{d_a}; \R^{d_{v_0}})$,$\kappa^{(0)} \in C(D \times D; \R^{d_{v_1} \times d_{v_0}})$,$b_0 \in C(D; \R^{d_{v_1}})$,$W_0 \in \R^{d_{v_1} \times d_{v_0}}$,$\kappa^{(1)} \in C(D' \times D; \R^{d_{v_2} \times d_{v_1}})$,以及 $\cQ \in C(\R^{d_{v_2}}; \R^{d_u})$。随后,我们可以用标准的前馈神经网络(feed-forward neural networks)(或其他任意方式)对连续函数 $\cP,\cQ,\kappa^{(0)},\kappa^{(1)},b_0$ 进行参数化,而矩阵 $W_0$ 则直接以其元素作为参数。参数向量 $\theta \in \R^p$ 便成为 $\cP,\cQ,\kappa^{(0)},\kappa^{(1)},b_0$ 的参数与 $W_0$ 的元素的拼接。随后,可通过标准的基于梯度的优化方法对 $\theta$ 进行优化以最小化损失函数。为实现该优化,损失函数中涉及的函数需进行离散化;但所学习到的参数可随后用于其他离散化方案。在第~\ref{sec:four_schemes} 节中,我们将讨论核函数参数化的不同选择、积分测度的选取,以及这些选择如何影响架构的计算复杂度。


\vspace{1cm}
\subsection{预处理(Preprocessing)} 通常,在输入函数 $a$ 中手动引入额外特征有助于促进学习过程。例如,\textcolor{red}{我们不直}\textcolor{red}{接将取值于 $\R^{d_a}$ 的向量场 $a$ 作为输入,而是使用取值于 $\R^{d + d_a}$ 的向量场 $(x, a(x))$}。通过引入恒等项(identity element),空间区域 $D$ 的几何信息便被直接嵌入到架构中。这使得神经网络可以直接访问问题中已知的信息,从而简化学习任务。我们在第~\ref{sec:numerics} 节的所有数值实验中均采用了这一思想。类似地,在学习一个光滑算子(smoothing operator)时,可能有益的做法是引入输入 $a$ 的光滑版本 $a_\epsilon$,例如通过高斯卷积(Gaussian convolution)获得。导数信息也可能具有价值,因此作为输入,可以考虑例如取值于 $\R^{d + 2d_a + dd_a}$ 的向量场 $(x, a(x), a_\epsilon(x), \nabla_x a_\epsilon (x))$。根据具体问题,还可以考虑许多其他可能性。

\vspace{1cm}
\subsection{离散化不变性与逼近性(Discretization Invariance and Approximation)}

根据离散化不变性定理~\ref{thm:discretizational_invariance} 以及通用逼近定理~\ref{thm:main_compact}、\ref{thm:cm_compact}、\ref{thm:measurable_approx}、\ref{thm:cm_measurable_approx}(其严格陈述见第~\ref{sec:approximation} 节),我们可以将神经算子所产生的总误差分解为\textcolor{red}{离散化误差}(discretization error)与\textcolor{red}{逼近误差}(approximation error)之和。具体而言,给定一个神经算子的有限维实现 $\hat{\G}_\theta : \R^{Ld} \times \R^{L d_a} \to \U$,对应于输入在某个 $L$ 点离散化下的表示,我们有
\[
\|\hat{\G}_\theta (D_L, a|_{D_L}) - \G^\dagger (a)\|_\U \leq \underbrace{\|\hat{\G}_\theta (D_L, a|_{D_L}) - \G_\theta (a) \|_\U}_{\text{离散化误差}} + \underbrace{\|\G_\theta (a) - \G^\dagger (a) \|_\U}_{\text{逼近误差}}.
\]
我们的逼近理论定理表明,我们可以选取参数 $\theta$ 使得逼近误差任意小;而离散化不变性定理则说明,我们可以选取足够精细的离散化(即足够大的 $L$),使得离散化误差也任意小。因此,借助一组与输入离散化无关的固定参数,可在计算机上实现的神经算子能够以任意精度逼近目标算子。

% \label{sec:discritizational_invariance}


% 基于上述构造及离散化不变性的定义,我们在下文中证明神经算子是离散化不变的深度学习模型。

%\begin{theorem}[离散化不变性与神经算子]
%\label{thm:discretizational_invariance_informal}
%以连续激活函数定义、并视为映射 $\A \to \U$ 的神经算子,是离散化不变的深度学习模型。
%\end{theorem}

% 定理~\ref{thm:discretizational_invariance_informal} 的严格陈述见第~\ref{sec:discritizational_invariance} 小节。该定理指出:当区域的离散化越来越精细时,作用于离散化域上的神经算子的行为将越来越接近连续算子的行为。证明见附录 \ref{proof:discretizational_invariance}。

% 为证明该定理,我们利用输入空间的紧性(compactness)论证存在一个有限函数字典,可逼近该空间中的任意函数。接着我们证明:对于任意给定的离散化,该字典中任一函数对应的积分算子的求和近似所产生的误差依赖于离散化的精细程度。因此,由于任意函数均可被该字典良好逼近,当使用求和近似积分时,随着离散化趋于精细,整体近似误差也随之减小。
