\vspace{1cm}
\subsection{Low-rank Neural Operator (LNO)(低秩神经算子)}
\label{sec:lowrank}

通过直接施加核函数 $\kappa$ 具有\textcolor{red}{张量积形式}的假设,我们得到了一个计算复杂度为 $\mathcal{O}(J)$ 的层。我们将这种构造称为低秩神经算子(Low-rank Neural Operator, LNO),因为\textcolor{red}{它等价于直接参数化一个有限秩算子}。我们首先假设 $\kappa : D \times D \to \R$ 是标量值函数,之后再推广到向量值情形。我们将核函数表示为
\[
\kappa(x,y) = \sum_{j=1}^r \varphi^{(j)}(x) \psi^{(j)}(y) \qquad \forall x,y \in D,
\]
其中函数 $\varphi^{(1)},\psi^{(1)},\dots,\varphi^{(r)},\psi^{(r)} : D \to \R$ 通常由两个神经网络 $\varphi, \psi : D \to \R^r$ 的分量给出,或者由一个单一的神经网络 $\Xi : D \to \R^{2r}$ 给出,该网络通过其参数将所有函数耦合在一起。在此定义下,并假设 $n=m=1$,则 \eqref{eq:onelayerlinear} 变为
\begin{align*}
    u(x) &= \int_D \sum_{j=1}^r \varphi^{(j)}(x) \psi^{(j)}(y) v(y) \: \text{d}y \\ 
    &= \sum_{j=1}^r \int_D \psi^{(j)} (y) v(y) \: \text{d}y \: \varphi^{(j)}(x) \\
    &= \sum_{j=1}^r \langle \psi^{(j)}, v \rangle \varphi^{(j)} (x),
\end{align*}
其中 $\langle \cdot, \cdot \rangle$ 表示 $L^2(D;\R)$ 内积。注意到这些内积的计算与评估点 $x \in D$ 无关,因此该方法的计算复杂度为 $\mathcal{O}(rJ)$,关于离散化点数呈线性。(有限秩指的是值域空间的维度有限。这里值域由 $\text{span}\{\varphi^{(1)},\dots,\varphi^{(r)}\}$ 给出。)

我们也可以将这种核函数的选择解释为直接在 $L^2(D;\R)$ 上参数化一个秩为 $r \in \N$ 的算子。事实上,我们有
\begin{equation}
    \label{eq:finiteranksvd}
    u = \sum_{j=1}^r (\varphi^{(j)} \otimes \psi^{(j)}) v,
\end{equation}
这恰好对应于将一个秩为 $r$ 的算子的奇异值分解(SVD)作用于函数 $v$。公式 \eqref{eq:finiteranksvd} 很自然地推广到向量值情形。假设 $m, n \geq 1$,且 $\varphi^{(j)}: D \to \R^m$、$\psi^{(j)} : D \to \R^n$($j=1,\dots,r$),那么 \eqref{eq:finiteranksvd} 定义了一个从 $L^2(D;\R^n)$ 到 $L^2(D;\R^m)$ 的算子,其计算形式为
\[
u(x) = \sum_{j=1}^r \langle \psi^{(j)}, v \rangle_{L^2(D;\R^n)} \varphi^{(j)}(x) \qquad \forall x  \in D.
\]
我们再次强调,这种参数化方式具有线性的计算复杂度。最后,我们注意到该方法也可以被理解为直接在核矩阵上施加一个秩为 $r$ 的结构。事实上,
\[
K = K_{Jr} K_{rJ},
\]
其中 $K_{Jr}$ 和 $K_{rJ}$ 分别是 $J \times r$ 和 $r \times J$ 的分块矩阵。

这种构造与 \cite{lu2019deeponet} 中讨论的 DeepONet 构造(见第~\ref{sec:deeponets} 节)类似,但其参数化方式在函数空间中具有一致性。

%虽然该方法具有线性计算复杂度,但它本质上是一种\textit{线性}逼近方法,可能无法有效捕捉解流形(solution manifold);详见第~\ref{sec:deeponets} 节的进一步讨论。

\postponed{据我所知,我们尚未真正从 DeVore/Dahmen 的解流形视角讨论过我们的学习问题;
或许可以改用我们已讨论过的 Bochner 范数来表述,
或者如果引入流形视角确实能带来新见解再加以介绍。但我并不确定这一点,因为实际上(在我们隐含假设每个输入对应唯一输出的前提下)它就是一个图(graph),而 Bochner 范数衡量的正是图或图之间的误差(我们的逼近本身也是一个图)。}

\iffalse
第二种方法基于算子 $\cK$(Fredholm 核)的低秩分解,称为低秩神经算子(Low-rank neural operator, LNO)。
令 $\U$ 为定义在 $\bar{D}$ 上、取值于 $\R$ 的函数构成的 Hilbert 空间。考虑一个紧致线性算子 $\cK: \U \to \U$。
该算子具有奇异值分解(SVD),因此可由其秩 $r$ 截断近似:
\[
\cK \approx \sum_{j=1}^r \varphi^{(j)} \otimes \psi^{(j)},
\]
其中 $\varphi^{(j)}, \psi^{(j)} \in \U$。
我们用两个因子神经网络来近似这两个因子函数 $\varphi^{(j)}, \psi^{(j)}: \R^{d+1} \to \R^{r \times d_v \times d_v}$(注意,无论秩 $r$ 多大,我们实际上只需要两个神经网络)。
该方法也可视为核积分的一种形式,通过假设核的可分离性,即
$\kappa\big(x,y, a(x), a(y)\big) = \sum_{j=1}^r \varphi^{(j)}\big(x, a(x)\big) \psi^{(j)}\big(y, a(y)\big)$,或简写为
\[
\kappa(x,y) = \sum_{j=1}^r \varphi^{(j)}(x) \psi^{(j)}(y).
\]
令 $\langle \cdot, \cdot \rangle$ 为 $L^2(D;\R)$ 内积,则
\begin{align*}
\bigl(\cK(a;\phi)v\bigr)(x) &= \int_D \kappa(x,y) v(y) \: dy \\
   &= \sum_{j=1}^r \left ( \int_D \psi^{(j)}(y) v(y) \: dy \right ) \varphi^{(j)}(x) \\
   & = \sum_{j=1}^r \langle \psi^{(j)}, v \rangle \varphi^{(j)}(x). 
\end{align*}

在离散形式下,低秩方法可视为对核矩阵 $K_{nn}$ 进行低秩分解:
\begin{align*}
    &K = \sum_j^r \varphi^{(j)} \otimes \psi^{(j)}\\
    &K_{nn} = K_{nr} K_{rn}
\end{align*}
此时,$\varphi^{(j)}$ 和 $\psi^{(j)}$ 为 $n$ 维向量,$K_{nr}$ 和 $K_{rn}$ 的每一列和每一行分别为 $\varphi^{(j)}$ 和 $\psi^{(j)}$。

\paragraph{谱分解与特征分解。}
由于 $K$ 是一个核,也可以使用特征值分解,即强制 $\varphi = \psi$。然而,这种约束会使神经网络的训练更加困难。实践中我们观察到,若固定 $\varphi = \psi$,神经算子的误差会更高。

\paragraph{秩与通道空间。}
实践中,低秩方法并不需要很高的秩。事实上,即使秩为 1 也能很好地近似核。这是因为在提升到更高维空间 $d_v$ 后,$\varphi^{(j)}, \psi^{(j)} \in \C^{n \times d_v \times d_v}$。
实践中,当 $d_v$ 足够大时,我们无需增加秩,只需设 $r=1$,因为高维空间本身已隐含了 $d_v \times d_v$ 的秩。

\paragraph{线性复杂度。}
可以看出,低秩方法的复杂度为 $O(n)$,其中 $n$ 为点的数量,因为我们只需分别对 $\varphi$ 和 $\psi$ 各评估 $n$ 次。
\fi


By directly imposing that the kernel \(\kappa\) is of a tensor product form, we obtain a layer with \(\mathcal{O}(J)\) computational complexity. We term this construction the Low-rank Neural Operator (LNO) due to its equivalence to directly parameterizing a finite-rank operator. We start by assuming \(\kappa : D \times D \to \R\) is scalar valued and later generalize to the vector valued setting. We express the kernel as
\[\kappa(x,y) = \sum_{j=1}^r \varphi^{(j)}(x) \psi^{(j)}(y) \qquad \forall x,y \in D\]
for some functions \(\varphi^{(1)},\psi^{(1)},\dots,\varphi^{(r)},\psi^{(r)} : D \to \R\) that are normally given as the components of two neural networks \(\varphi, \psi : D \to \R^r\) or a single neural network \(\Xi : D \to \R^{2r}\) which couples all functions through its parameters. With this definition, and supposing that \(n=m=1\), we have that \eqref{eq:onelayerlinear} becomes
\begin{align*}
    u(x) &= \int_D \sum_{j=1}^r \varphi^{(j)}(x) \psi^{(j)}(y) v(y) \: \text{d}y \\ 
    &= \sum_{j=1}^r \int_D \psi^{(j)} (y) v(y) \: \text{d}y \: \varphi^{(j)}(x) \\
    &= \sum_{j=1}^r \langle \psi^{(j)}, v \rangle \varphi^{(j)} (x)
\end{align*}
where \(\langle \cdot, \cdot \rangle\) denotes the \(L^2(D;\R)\) inner product. Notice that the inner products can be evaluated independently of the evaluation point \(x \in D\) hence the computational complexity of this method is \(\mathcal{O}(rJ)\) which is linear in the discretization. 

We may also interpret this choice of kernel as directly parameterizing a rank \(r \in \N\) operator on \(L^2(D;\R)\). Indeed, we have
\begin{equation}
    \label{eq:finiteranksvd}
    u = \sum_{j=1}^r (\varphi^{(j)} \otimes \psi^{(j)}) v 
\end{equation}
which corresponds preceisely to applying the SVD of a rank \(r\) operator to the function \(v\). Equation \eqref{eq:finiteranksvd} makes natural the vector valued generalization. Assume \(m, n \geq 1\) and \(\varphi^{(j)}: D \to \R^m\) and \(\psi^{(j)} : D \to \R^n\) for \(j=1,\dots,r\) then, \eqref{eq:finiteranksvd} defines an operator mapping 
\(L^2(D;\R^n) \to L^2(D;\R^m)\) that can be evaluated as
\[u(x) = \sum_{j=1}^r \langle \psi^{(j)}, v \rangle_{L^2(D;\R^n)} \varphi^{(j)}(x) \qquad \forall x  \in D.\]
We again note the linear computational complexity of this parameterization. Finally, we observe that this method can be interpreted as directly imposing a rank \(r\) structure on the kernel matrix. Indeed,
\[K = K_{Jr} K_{rJ}\]
where \(K_{Jr}, K_{rJ}\) are \(J \times r\) and \(r \times J\) block matricies respectively. 
This construction is similar to the DeepONet construction of \cite{lu2019deeponet} discussed in Section~\ref{sec:deeponets}, but parameterized to be consistent in function space.


%While this method enjoys a linear computational complexity, it also constitutes a \textit{linear} approximation method which may not be able to effectively capture the solution manifold; see Section~\ref{sec:deeponets} for further discussion. 


\postponed{We have not really discussed our learning problem using the
solution manifold perspective of DeVore/Dahmen as far as I recall;
maybe rephrase in terms of Bochner norm that we have discussed,
or introduce that manifold perspective if it adds something. I am not convinced it does because it  actually is a graph (if output unique
for every input which we implicitly assume) and Bochner
measures the graph or errors between graphs (our approximation
is also a graph).}

\iffalse
The second method bases on the low-rank decomposition of operator $\cK$ (the Fredholm kernel), referred as Low-rank neural operator (LNO).
let \(\U\) be a Hilbert space of functions defined on \(\bar{D}\) taking values in \(\R\). Consider a compact, linear operator \(\cK: \U \to \U\).
This operator admits a singular value decomposition (SVD) and can therefore be approximated by its rank \(r\) truncation:
\[\cK \approx \sum_{j=1}^r \varphi^{(j)} \otimes \psi^{(j)}\]
for some \(\varphi^{(j)}, \psi^{(j)} \in \U\).  
We approximate the two factor functions by two factor neural networks $\varphi^{(j)}, \psi^{(j)}: \R^{d+1} \to \R^{r \times d_v \times d_v}$ (notice we have only two neural networks matter the rank).
This approach can also be seen as kernel integration by assuming separability of the kernel, in particular,
$\kappa\big(x,y, a(x), a(y)\big) = \sum_{j=1}^r \varphi^{(j)}\big(x, a(x)\big) \psi^{(j)}\big(y, a(y)\big)$, or simply
\[\kappa(x,y) = \sum_{j=1}^r \varphi^{(j)}(x) \psi^{(j)}(y).\]
Let \(\langle \cdot, \cdot \rangle\) be the \(L^2(D;\R)\) inner-product. Then
\begin{align*}
\bigl(\cK(a;\phi)v\bigr)(x) &= \int_D \kappa(x,y) v(y) \: dy \\
   &= \sum_{j=1}^r \left ( \int_D \psi^{(j)}(y) v(y) \: dy \right ) \varphi^{(j)}(x) \\
   & = \sum_{j=1}^r \langle \psi^{(j)}, v \rangle \varphi^{(j)}(x). 
\end{align*}

In the discretized format, the low-rank methods can be viewed as an low-rank decomposition on the kernel matrix $K_{nn}$.
\begin{align*}
    &K = \sum_j^r \varphi^{(j)} \otimes \psi^{(j)}\\
    &K_{nn} = K_{nr} K_{rn}
\end{align*}
In this case, $\varphi^{(j)}$ and $\psi^{(j)}$ are $n$-dimensional vector, and $K_{nr}, K_{rn}$ has each column and row to be $\varphi^{(j)}$ and $\psi^{(j)}$ respectively.


\paragraph{Spectral decomposition and eigen decomposition.}
Since $K$ is a kernel, it is possible to use eigenvalue decomposition by forcing $\varphi = \psi$. However, this constrains will make training of the neural network more difficult. In practice, we observe fixing $\varphi = \psi$ the neural operator will have a higher error rate.

\paragraph{Rank and channel space.}
In practice, the low-rank method does not require a high rank. Indeed it can approximate the kernel even with rank one. This is when we lift to higher dimension $d_v$, $\varphi^{(j)}, \psi^{(j)} \in \C^{n \times d_v \times d_v}$. 
In practice, when $d_v$ is sufficiently large, we don't need additional rank, but simply set $r=1$, because the higher dimension space induces implicit rank $d_v \times d_v$.

\paragraph{Linear complexity.}
It can be seen that the low-rank method has complexity $O(n)$, where $n$ is the number f points, since we only need to evaluate $\varphi$ and $\psi$ each for $n$ times.
\fi
